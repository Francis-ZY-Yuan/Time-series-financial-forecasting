\documentclass{article}
\usepackage[utf8]{inputenc}

\title{KPMG 4 First Progress Report}
\author{Names and UNI here}
\date{October 2022}

\begin{document}

\maketitle

\section{Introduction}


\section{Methodology}
\subsection{Problem Formulation and Modeling Approach}
 Our  \ final \ goal \ of \ this \ project \ is \ to \ built \ a \ model \ with \ good \ performance \ in \ multi-variate \ financial \ time \ series \ forecasting. \ To \ achieve \ this \ goal,  \ we \ start \ from \ the \ simplest \ model \ and \ try \ to \ scale \ up \ the \ problem. \\ \\
 In \ the \ first \ stage(current \ stage) \ we \ try \ to \ build \ univariate \ classic \ ARIMA \ models \ for \ each \ individual \ features. \\
For \ each \ variable \ x \\
Feature: $(x_1,x_2,…x_t)$ \\
Target: $(x_{t+1},…x_{t+k})$ \\ \\In  \ the \ second \ stage, \ we \ would \ want \ to \ build \ a \ multivariate \ regression \ model \ with \ some \ classic \ ensemble \ learning \ models \ like \ randomforest \ or \ xgboost.\\Let \ the \ multivariate \ feature \ be \ $\vec{x}_{t} =(x_{t1},x_{t2},…x_{tm}))$ \ at \ time \ t, \ for \ each \ feature \ $x_{m}$ \ we \ need \ to \ build \ an \ individual \ model, that is, for the $i_{th}$ dimension\\
Feature: $(\vec{x}_1,\vec{x}_2,…\vec{x}_t)$ \\
Target: $x_{(t+1)i}$ \\ \\In \ the \ final \ stage, \ we \ would \ like \ to \ explore \ the \ usage \ of \ deep \ learning, \ and \ see \ if \ it \ can \ be \ more \ powerful \ than \ traditional \ approaches. \\Feature: \ $(\vec{x}_1,\vec{x}_2,…\vec{x}_t)$ \\Target: \ $(\vec{x}_{t+1},\vec{x}_{t+2},…\vec{x}_{t+k})$

\subsection{Related Literature}

Currently \ there \ are \ two \ papers \ we \ would \ like \ to \ reference \ from.\\
1, \ A \ Multi-Faceted \ Approach \ to \ Large \ Scale \ Financial \ Forecasting \\ The \ paper \ has \ a \ very \ similar \ approach \ as \ we \ do, \ starting \ from \ a \ simple \ baseline \ and \ gradually \ scale \ up, \ which \ can \ serve \ as \ a \ nice \ guideline. \\ \\2, A \ Transformer-based \ Framework \ for \ Multivariate \ Time \ Series \ Representation \ Learning \\Transformer \ is \ a \ powerful \ technique \ which \ is \ widely \ used \ in \ representation \ learning \ like \ NLP. \ We \ would \ like \ to \ explore \ its \ usage \ in \ structured \ data \ regression \ problem.  
\section{Results}
\subsection{Feature Extraction}
\subsection{Uni-variate Modeling Results}
\end{document}
